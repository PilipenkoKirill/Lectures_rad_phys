%!TEX program = xelatex.exe
\documentclass[14pt,a4paper]{article}
\usepackage{cmap} %Улучшает поиск по pdf документу
\usepackage[pdftex,
    pdfauthor={К.С.~Пилипенко},
    pdftitle={Список тем по радиационной физике},
    pdfsubject={The Subject},
    pdfkeywords={Первое ключевое слово, второе ключевое слово},
    pdfproducer={LuaLatex with hyperref},
    pdfcreator={Lualatex},
    hidelinks
]{hyperref}
%%%%%%%%%%%%%%Пользовательские команды%%%%%%%%%
\usepackage{latexsym,amsmath,amssymb,amsbsy,graphicx}
\usepackage{icomma}
\usepackage[version=4]{mhchem} % the canonical chemistry package (example: \ce{^{32}_{15}P})
\usepackage{graphicx}
\graphicspath{{images/}}
\DeclareGraphicsExtensions{.pdf,.png,.jpg}
%%%%%%%%%%%%%%%%%%%%%%%%Оформление по ГОСТУ
\usepackage{fontspec}
\setmainfont[Renderer=Basic,Ligatures={TeX}]{Times New Roman}
\usepackage[english,russian]{babel} %Поддержка русской локализации
\usepackage[14pt]{extsizes} % для того чтобы задать нестандартный 14-ый размер шрифта
\usepackage{indentfirst} %Задаёт отступ самого первого абзаца
\setlength\parindent{1.25cm}
\usepackage[a4paper, left=3cm, top=1.5cm, right=1.5cm, bottom=2cm]{geometry}
\usepackage{setspace}
%\sloppy %Выравнивание текст по ширине и решение проблемы переполнением строки
\onehalfspacing %Полуторный интервал
\usepackage{mathtext} % русские буквы в формулах
\usepackage{caption} %заголовки плавающих объектов
\captionsetup[figure]{name=Рис.} % меняет название рисунков на русское
%%%%%%%%%%%%%%%%%%%%%%%%%%%%%
\title{Список тем по радиационной физике}
\author{\href{mailto:www-kirill.pilipenko@yandex.ru}{К.С.~Пилипенко}} %Через \and можно добавить ещё авторов
\date{\selectlanguage{russian}\today}
\begin{document}
\maketitle
    \begin{enumerate}
        \item История развития радиационной биофизики. Пионеры радиобиологии. Открытие закона радиочувствительности клеток.
        \item Биологические эффекты малых доз ионизирующей радиации. Радиационный гормезис. Радиационно-индуцированный адаптивный ответ.
        \item Виды ионизирующего излучения, их получение и практическое использование.
        \item Механизмы гибели и процессы восстановления клеток от радиационного поражения. Повреждения и процессы восстановления ДНК в облученной клетке. 
        \item Продолжительность жизни млекопитающих в зависимости от дозы облучения. Лучевая болезнь человека и ее стадии. 
        \item Факторы, модифицирующие лучевое поражение: радиопротекторы и радиосенсибилизаторы, их химическая природа и биологическое действие.
        \item Радиационно-индуцированная нестабильность генома и ее биологическое значение.
        \item Ионизация в тканях косвенно ионизирующими частицами.
        \item Методы дозиметрии. Приборы для регистрации ионизирующих излучений.
        \item Использование радиоактивных изотопов в биологии и медицине. Радиоактивные фар-мацевтические препараты.
        \item Оборудование классической дистанционной лучевой терапии. Гамма-установки с ра-диоактивным источником. Медицинские линейные ускорители. Томотерапия. Гамма-нож. Кибернож.
        \item Радиационные синдромы: костномозговой, желудочно-кишечный, церебральный. 
        \item Радиационная безопасность в РФ и за рубежом. Средняя годовая доза облучения жи-телей России.
        \item Лучевая терапия. Методы лучевой терапии. Предлучевой, лучевой и постлучевой пе-риоды. Осложнения и борьба с ними. 
        \item Эффект Черенкова-Вавилова. Черенковский детектор.
        \item Оборудование контактной лучевой терапии. Аппараты брахитерапии.
        Аппараты интраоперационной лучевой терапии.
    \end{enumerate}
\end{document}